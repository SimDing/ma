%% LaTeX2e class for student theses
%% sections/abstract_de.tex
%% 
%% Karlsruhe Institute of Technology
%% Institute for Program Structures and Data Organization
%% Chair for Software Design and Quality (SDQ)
%%
%% Dr.-Ing. Erik Burger
%% burger@kit.edu
%%
%% Version 1.3.6, 2022-09-28

\Abstract
Die Automatisierung von Testprozessen ist ein wichtiger Schritt zur Verbesserung der Softwarequalität und zur Minimierung von Fehlern.
Automatisierte Tests können auch Zeit und Kosten sparen, indem sie eine schnelle und effiziente Identifizierung und Behebung von Fehlern ermöglichen.
Die Implementierung von Tests, die ausschließlich auf einer grafischen Benutzeroberfläche basieren, kann jedoch oft zeitaufwändig und teuer sein.
Automatisiertes exploratives Testen kann verwendet werden, um Bedingungen zu finden, die Abstürze auslösen, kann jedoch sehr langsam sein.
Tests mit natürlicher Sprache können eine breitere Kategorie von Tests abdecken, einschließlich funktionaler Tests und Regressionstests.
Sie können entweder manuell von einem Tester ausgeführt oder als Skript spezifiziert und mit einem Tool wie Selenium ausgeführt werden.
Obwohl solche automatisierten GUI-Tests wünschenswert sind, müssen sie dennoch programmiert und gewartet werden.
Um diesen Prozess zu vereinfachen, wäre es vorteilhaft, Tests in natürlicher Sprache zu definieren und automatisch ohne menschliches Eingreifen auszuführen.
Das Ziel dieser Arbeit besteht darin, dies zu versuchen, indem natürlichsprachliche Testfälle und eine strukturierte Darstellung der grafischen Benutzeroberfläche als Eingabe für ein großes Sprachmodell (LLM) verwendet werden, um zu entscheiden, welche Benutzeraktion als nächstes simuliert werden soll.